%まえがき
新型コロナウイルスの世界最初の症例が確認されてから、1年以上が経過した2021年1月現在、日本のみならず世界中で、いまなおこの感染症が猛威を振るっている。この1年の間、新型コロナウイルスへの感染予防対策の一環として、巷で叫ばれてきたのが、いわゆる「3密」の回避の重要性である。新型コロナウイルスの特性として、把握されている事実としては、「一般的な状況における感染経路は飛沫感染、接触感染であり、空気感染は起きていないと考えられ、閉鎖空間において近距離で多くの人と会話する等の一定の環境下であれば、咳やくしゃみ等がなくても感染を拡大させるリスクがある\cite{housin}」というものである。新型コロナウイルスのこういった特性から、私たちがこの感染症から自らの身を守るために取るべき対策として、まず考えるべきことが、「密閉」「密集」「密接」の、3つの密を避けるという対策であり、こういった対策を講じることが、自分自身だけでなく、身の回りの人を感染症から守ることにつながると考えられる。

こういった点に着目し、新型コロナウイルスに対する感染症予防対策の一環として用いる、感染症予防サポートシステムの開発を本研究の目標とした。本研究で開発するシステムでは、感染症予防対策の基準を定める役割、感染症予防対策のルールを守ってもらうよう働きかける役割を担い、3密回避を利用者に意識付け、部屋の利用者が適切に安全な環境づくりができているかをモニタリングするシステムの構築を最終目標とし、開発に着手した。また本研究には、4人のメンバーを1つのチームとし、取り組むこととなった。

現在世の中では、新型コロナウイルスの世界的な流行を受けて、感染症予防のための既存製品が注目を集めるようになっているほか、新たな製品やシステムの開発に、世界中の多くの企業が取り組んでいるという状況にある。そのような状況の中で、本研究において実現するシステムが、他の感染予防のための製品やシステムと比較した際に特異といえる要素こそが、3密回避を実現する機能を有しているという点にある。部屋の換気状態のみを監視し、密閉状態を防ぐ機能を有するシステムは既に数多く存在している。一方で本研究では、換気状態、部屋の広さ、および室内の滞在人数をもとに、部屋の特性に応じた感染症予防対策のルールを設定できるシステムを設計し、従来の感染症予防のためのシステムと比較し、より新型コロナウイルスのような特徴を持つ感染症への、感染予防に特化した機能を有するシステムの構築に取り組んでいる。このように、私たちはこの感染症の特性を理解したうえで、講じられるべき対策として、「3密回避」に重点を置き、IoTシステムにおいて実現できる方法を議論した。

具体的には、AIエッジ向けコンピュータとして知られるJetsonシリーズのJetson nanoと、無線マイコンモジュールTWELITEをそれぞれエッジサーバ、センサデバイスとして用い、室内に滞在している人の数や二酸化炭素濃度などの室内環境値の測定と、その測定値をもとにした環境評価、部屋の利用者へのモニタリング結果の通知という機能を持つシステムを実現目標として定めた。また、室外からも室内の環境に関する情報を受け取ることができるよう、部屋への入室の危険度を表示する機能も実現し、室内外から感染予防のための情報を受け取ることができる。

室内環境のモニタリングにおいては、室内の画像をもとにした人数の推定と、二酸化炭素濃度センサや温湿度センサを用いた環境値の測定を行っている。室内画像をもとにした人数の推定を行うデバイスに関しては、よりリアルタイム性のあるモニタリングの実現のため、処理時間の短縮化が求められるために、GPUを搭載し、高速な浮動小数点演算を可能とするJetson nanoを選定した。二酸化炭素濃度センサや温湿度センサを用いた環境値の測定に関しては、デバイスの取り付け場所が、電源の供給方法による制約を受けず、なおかつ比較的低い消費電力での稼働を可能とするデバイスとして、単4乾電池2本で動作し、ワイヤレスセンサーネットワークの構築に適した無線規格であるIEEE802.15.4を採用し、低消費電力での無線通信を可能にする無線マイコンモジュールとしてTWELITEを選定し、室外に設置する部屋への入室の危険度を表示するデバイスに関しても、同じくTWELITEを選定した。

今回のシステムの開発は、V字モデルに従って行い、要求分析・基本設計・詳細設計には、UML(Unified Modeling Language)を用いている。

また、本論文の構成は以下の通りである。

第1章では研究背景について述べた。第2章では本研究で用いる用語や研究方針について述べる。第3章ではUML図作成を中心としたシステムの設計について述べる。第4章では、システムの実装と検証結果について述べ、第5章では実装に関する情報と、検証による評価・考察を示し、第6章では本研究のまとめを行う。  


 

%第4-1章
\section{実装}
ここから、3章の設計に基づいて行った実装について説明する。

3章でも述べたが、本システムの開発において私が担当したのは、エッジサーバとしての役割を持つJetson nano側の機能実現である。ここで、実際に用いたJetson nanoの仕様を表\ref{jetsonnano}に示す。

\begin{table}[H]
	\begin{center}
		\caption{Jetson nano仕様}
		\begin{tabular}{|c||c|} \hline
			OS & Ubuntu18.04\\ \hline
			CPU & クアッドコア ARM A57 @ 1.43 GHz \\ \hline
			GPU & 128 コア Maxwell\\ \hline
			メモリ & 4 GB 64 ビット LPDDR4 25.6 GB/秒\\ \hline
		\end{tabular}
		\label{jetsonnano}
	\end{center}
\end{table}

また自身は、センサデバイスを用いた環境値測定および人数推定による環境監視と、環境監視から得られたデータの分析を繰り返すメインプログラム、センサデバイスへのモニタリング開始命令とデータ受信、室外デバイスへのLED 点灯命令送信を行う、デバイス類との通信関係のプログラム、受信データや分析結果を記録・管理するため、データベースを操作するプログラムを作成した。担当箇所が、システムの中心となる部分であったことから、他のメンバーの進捗に合わせ統合も並行して進めるなど、全体の進捗も気にかけながら作業を進めた。

自身が担当し作成したこれらのプログラムは、主な開発言語をpythonとして実装している。もう少し詳細にこれらプログラムの内容について説明する。

\subsection*{メインプログラム}
このプログラムは、エッジサーバ側で行う処理全体を動作させるプログラムで、実行後にはまず、現在時刻が8時より早い、あるいは20時以降である場合にはスリープに入り、当日または翌朝の8時からモニタリングを開始する。初回起動時が8時以降かつ20時より早い時間帯であれば、20時までのモニタリングをすぐに実行できる。

モニタリング開始前には別途作成した、センサデバイスとの通信用プログラムを別スレッドで開始し、以降この別スレッドプログラムが存在する限りモニタリングを繰り返す。モニタリング機能は、まずセンサデバイスからの受信データを管理するデータベースに、接続中のデバイス台数分のデータ増加があるかを確認することで、データ更新があるまでの待機を行う。データ更新があれば、データ分析が可能になるため、Webカメラからの室内画像の取得を行い、人数推定機能担当者が作成した、人数推定用プログラムを呼び出す。このプログラムから、画像内に含まれる人の数を受け取ると、データベースから過去15分間のデータに相当する、最新の5回分の各デバイスのデータを取り出し、分析を行う。ただし、取得するデータは直近21分以内のデータに限定しているため、センサデバイスが電池切れなどで一時通信を行わず、再び復帰した場合に最新5 回分のデータが過度に古いデータにならぬよう対策を行っている。

部屋の二酸化炭素濃度に関しては、最新の5回分の各デバイスのデータについて最小値を導き、全デバイスの最小値を比較し、最大となるものを代表値とすることで、室内のうち最も換気状態の悪い箇所が15分間連続し、最低でどのような値をとっていたかを導き出している。

その後、この代表値から部屋の警戒レベルを求め、室内にいる人数が、導かれた警戒レベルで制限されている滞在可能人数に適合しているかを確認し、感染リスクを評価する。ここまでの内容を、夜間スリープに入る20時まで繰り返すのがメインプログラムの基本的な動作となる。以上のように、3章で示した設計内容に順じ、エッジサーバ側の機能を実装した。

\subsection*{各デバイスとの通信用プログラム}
このプログラムは、センサデバイス・室外デバイスとの通信を行うプログラムで、開始されるとまず、センサデバイスにセンサ値取得開始命令を送る。センサデバイスは省電力化の関係から、開始命令受信後は受信待ち状態に基本的には入らないことから、何分おきに、何回分のデータ取得・送信を行うかを開始命令として送信する。以降は、センサデバイスからのデータ受信と、モニタリング結果をもとにした、室外デバイスへのLED点灯命令を送る処理を繰り返す。

\subsection*{データベース操作プログラム}
このプログラムは、python用のmysql connecterを用いて、リレーショナルデータベース管理システムMySQLを操作する。メインプログラムから本プログラムのメソッドが呼び出され、データ数や時刻を引数として受け取り、データの取り出しなどの操作を仲介する。


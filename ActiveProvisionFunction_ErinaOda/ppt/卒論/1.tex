%まえがき

昨今,新型コロナウィルス感染症が世界的に流行し,人々の生活に大きく影響を及ぼしている.そして感染症を予防するために,感染リスクの高い場面を避けることが呼びかけられている.
新型コロナウィルス感染症は,一般的に飛沫感染および接触感染により感染し,密閉,密集,密接の3つの密によって感染リスクが高まると言われている\cite{qa}.感染リスクの高い「密閉」空間をつくらないために,換気が一つの方法として挙げられる.しかし換気を行った結果,どのように空気環境が改善されたかは一目でわかる訳ではない.ビル管理法における空気環境の基準として,浮遊粉じんの量,一酸化炭素の含有率,二酸化炭素の含有率,温度,湿度,気流,ホルムアルデヒドの量が定義されている\cite{kanki}.
愛媛大学工学部附属社会基盤iセンシングセンターの実験によると,部屋の換気の指標として二酸化炭素濃度の計測が有効と思われると結果が出ている\cite{isence}.
室内の二酸化炭素濃度を測定して,危険な状態になる前に室内の利用者や管理者へ換気する注意喚起できるシステムがあれば,室内の利用者が感染リスクを回避することができる.
また,室外の利用予定者(これから施設に入ろうとする人)に対しては,現時点の施設の利用状態を把握できれば,入室することを控えることで,密閉の未然防止につながる.

本研究では,利用者の多い大学の講義室での3密状態を避けるために,IoT センシング技術を用いて,室内の二酸化炭素の濃度値,温湿度値,在室人数などの室内環境をリアルタイムにモニタリングし,利用者と管理者に感染リスクを注意喚起する「感染症予防サポートシステム」を開発する.

本システムを開発するために,本研究では,4人のメンバー(伊藤大輝,稲田一輝,小田恵吏奈,掛水誠矢)を1つのチームとし,開発を進める.開発チーム内の役割分担として,小田はセンサーとカメラで収集した施設内の環境情報(在室人数,入室できる人数,二酸化炭素濃度水準,換気状態,感染リスクなどの情報)を利用者対象別に積極的に提供する「室内環境情報の能動的提供機能」を開発することを目標とする.具体的には,無線マイコンモジュールTWELITE をエッジデバイスとして用いて,エッジサーバ(掛水,稲田,伊藤担当)で計測した室内に滞在している人の数と二酸化炭素濃度などの室内環境値をもとにした環境評価情報(部屋への入室の危険度)を部屋の利用者(室内の利用者と利用予定者)へモニタリング結果を通知する機能を開発する.

システムの開発は,V字モデルに従って,グループで議論しながら共同で行った.また,共同開発中に,グループ内での考え方・進め方に矛盾が生じないように,UML(Unified Modeling Language)を用いて,システムの要求分析,基本設計,詳細設計を行った.

本論文の構成は下記のとおりである.第2章では本論文で用いる用語や研究方針について述べる.第3章では本システム全体の概要とV字モデルに従った本システムの設計について述べる.第4章では,デバイスの実装と検証結果について述べる.第5章では実装・検証した本システムの評価を行い,考察を示す.第6章では本研究のまとめを行う.
